% Options for packages loaded elsewhere
\PassOptionsToPackage{unicode}{hyperref}
\PassOptionsToPackage{hyphens}{url}
\PassOptionsToPackage{dvipsnames,svgnames,x11names}{xcolor}
%
\documentclass[
  letterpaper,
  DIV=11,
  numbers=noendperiod]{scrartcl}

\usepackage{amsmath,amssymb}
\usepackage{lmodern}
\usepackage{iftex}
\ifPDFTeX
  \usepackage[T1]{fontenc}
  \usepackage[utf8]{inputenc}
  \usepackage{textcomp} % provide euro and other symbols
\else % if luatex or xetex
  \usepackage{unicode-math}
  \defaultfontfeatures{Scale=MatchLowercase}
  \defaultfontfeatures[\rmfamily]{Ligatures=TeX,Scale=1}
\fi
% Use upquote if available, for straight quotes in verbatim environments
\IfFileExists{upquote.sty}{\usepackage{upquote}}{}
\IfFileExists{microtype.sty}{% use microtype if available
  \usepackage[]{microtype}
  \UseMicrotypeSet[protrusion]{basicmath} % disable protrusion for tt fonts
}{}
\makeatletter
\@ifundefined{KOMAClassName}{% if non-KOMA class
  \IfFileExists{parskip.sty}{%
    \usepackage{parskip}
  }{% else
    \setlength{\parindent}{0pt}
    \setlength{\parskip}{6pt plus 2pt minus 1pt}}
}{% if KOMA class
  \KOMAoptions{parskip=half}}
\makeatother
\usepackage{xcolor}
\setlength{\emergencystretch}{3em} % prevent overfull lines
\setcounter{secnumdepth}{-\maxdimen} % remove section numbering
% Make \paragraph and \subparagraph free-standing
\ifx\paragraph\undefined\else
  \let\oldparagraph\paragraph
  \renewcommand{\paragraph}[1]{\oldparagraph{#1}\mbox{}}
\fi
\ifx\subparagraph\undefined\else
  \let\oldsubparagraph\subparagraph
  \renewcommand{\subparagraph}[1]{\oldsubparagraph{#1}\mbox{}}
\fi


\providecommand{\tightlist}{%
  \setlength{\itemsep}{0pt}\setlength{\parskip}{0pt}}\usepackage{longtable,booktabs,array}
\usepackage{calc} % for calculating minipage widths
% Correct order of tables after \paragraph or \subparagraph
\usepackage{etoolbox}
\makeatletter
\patchcmd\longtable{\par}{\if@noskipsec\mbox{}\fi\par}{}{}
\makeatother
% Allow footnotes in longtable head/foot
\IfFileExists{footnotehyper.sty}{\usepackage{footnotehyper}}{\usepackage{footnote}}
\makesavenoteenv{longtable}
\usepackage{graphicx}
\makeatletter
\def\maxwidth{\ifdim\Gin@nat@width>\linewidth\linewidth\else\Gin@nat@width\fi}
\def\maxheight{\ifdim\Gin@nat@height>\textheight\textheight\else\Gin@nat@height\fi}
\makeatother
% Scale images if necessary, so that they will not overflow the page
% margins by default, and it is still possible to overwrite the defaults
% using explicit options in \includegraphics[width, height, ...]{}
\setkeys{Gin}{width=\maxwidth,height=\maxheight,keepaspectratio}
% Set default figure placement to htbp
\makeatletter
\def\fps@figure{htbp}
\makeatother




























\ifLuaTeX
  \usepackage{selnolig}  % disable illegal ligatures
\fi
\IfFileExists{bookmark.sty}{\usepackage{bookmark}}{\usepackage{hyperref}}
\IfFileExists{xurl.sty}{\usepackage{xurl}}{} % add URL line breaks if available
\urlstyle{same} % disable monospaced font for URLs
\hypersetup{
  pdftitle={Hello Callout!},
  colorlinks=true,
  linkcolor={blue},
  filecolor={Maroon},
  citecolor={Blue},
  urlcolor={Blue},
  pdfcreator={LaTeX via pandoc}}

\title{Hello Callout!}
\author{}
\date{}

\begin{document}
\maketitle


\hypertarget{overview-of-callouts}{%
\subsection{Overview of callouts}\label{overview-of-callouts}}

\begin{tcolorbox}[enhanced jigsaw, leftrule=.75mm, title=\textcolor{quarto-callout-warning-color}{\faExclamationTriangle}\hspace{0.5em}{Warning}, coltitle=black, left=2mm, colback=white, bottomtitle=1mm, opacitybacktitle=0.6, opacityback=0, colframe=quarto-callout-warning-color-frame, colbacktitle=quarto-callout-warning-color!10!white, breakable, toptitle=1mm, bottomrule=.15mm, titlerule=0mm, arc=.35mm, rightrule=.15mm, toprule=.15mm]

Callouts provide a simple way to attract attention, for example, to this warning.

\end{tcolorbox}

\begin{tcolorbox}[enhanced jigsaw, leftrule=.75mm, title=\textcolor{quarto-callout-important-color}{\faExclamation}\hspace{0.5em}{}, coltitle=black, left=2mm, colback=white, bottomtitle=1mm, opacitybacktitle=0.6, opacityback=0, colframe=quarto-callout-important-color-frame, colbacktitle=quarto-callout-important-color!10!white, breakable, toptitle=1mm, bottomrule=.15mm, titlerule=0mm, arc=.35mm, rightrule=.15mm, toprule=.15mm]

Danger, callouts will really improve your writing.

\end{tcolorbox}

\begin{tcolorbox}[enhanced jigsaw, leftrule=.75mm, title=\textcolor{quarto-callout-note-color}{\faInfo}\hspace{0.5em}{Note}, coltitle=black, left=2mm, colback=white, bottomtitle=1mm, opacitybacktitle=0.6, opacityback=0, colframe=quarto-callout-note-color-frame, colbacktitle=quarto-callout-note-color!10!white, breakable, toptitle=1mm, bottomrule=.15mm, titlerule=0mm, arc=.35mm, rightrule=.15mm, toprule=.15mm]

Note that there are five types of callouts, including: \texttt{note}, \texttt{tip}, \texttt{warning}, \texttt{caution}, and \texttt{important}.

\end{tcolorbox}

\begin{tcolorbox}[enhanced jigsaw, leftrule=.75mm, title=\textcolor{quarto-callout-tip-color}{\faLightbulb}\hspace{0.5em}{}, coltitle=black, left=2mm, colback=white, bottomtitle=1mm, opacitybacktitle=0.6, opacityback=0, colframe=quarto-callout-tip-color-frame, colbacktitle=quarto-callout-tip-color!10!white, breakable, toptitle=1mm, bottomrule=.15mm, titlerule=0mm, arc=.35mm, rightrule=.15mm, toprule=.15mm]

This is an example of a callout with a caption.

\end{tcolorbox}

\begin{tcolorbox}[enhanced jigsaw, leftrule=.75mm, title=\textcolor{quarto-callout-tip-color}{\faLightbulb}\hspace{0.5em}{}, coltitle=black, left=2mm, colback=white, bottomtitle=1mm, opacitybacktitle=0.6, opacityback=0, colframe=quarto-callout-tip-color-frame, colbacktitle=quarto-callout-tip-color!10!white, breakable, toptitle=1mm, bottomrule=.15mm, titlerule=0mm, arc=.35mm, rightrule=.15mm, toprule=.15mm]

This is an example of a callout with a caption containing special formatting and characters.

\end{tcolorbox}

\begin{tcolorbox}[enhanced jigsaw, leftrule=.75mm, title=\textcolor{quarto-callout-caution-color}{\faFire}\hspace{0.5em}{}, coltitle=black, left=2mm, colback=white, bottomtitle=1mm, opacitybacktitle=0.6, opacityback=0, colframe=quarto-callout-caution-color-frame, colbacktitle=quarto-callout-caution-color!10!white, breakable, toptitle=1mm, bottomrule=.15mm, titlerule=0mm, arc=.35mm, rightrule=.15mm, toprule=.15mm]

This is an example of a `collapsed' caution callout that can be expanded by the user. You can use \texttt{collapse="true"} to collapse it by default or \texttt{collapse="false"} to make a collapsible callout that is expanded by default.

\end{tcolorbox}

\begin{tcolorbox}[enhanced jigsaw, leftrule=.75mm, colframe=quarto-callout-color-frame, left=2mm, breakable, colback=white, toprule=.15mm, arc=.35mm, opacityback=0, rightrule=.15mm, bottomrule=.15mm]

\textbf{Exercise}\vspace{2mm}

You can also use a plain \texttt{callout} class to get a simple callout treatment.

\end{tcolorbox}



\end{document}
